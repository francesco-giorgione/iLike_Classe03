
% Preamble
\documentclass[a4paper, 12pt]{report}
\usepackage[T1]{fontenc}
\usepackage[utf8]{inputenc}
\usepackage[italian]{babel}
\usepackage{lipsum}
\usepackage{url}
\usepackage{array}
\usepackage{color}
\usepackage{colortbl}
\usepackage{adjustbox}
\usepackage[dvipsnames]{xcolor}
\usepackage{graphicx}
\usepackage[colorlinks = true,
    linkcolor = darkgray,
    urlcolor  = darkgray,
    citecolor = white,
    anchorcolor = darkgray]{hyperref}

\makeatletter
    \def\@makechapterhead#1{%
        \vspace*{1\p@}%
        {\parindent \z@ \raggedright
        \normalfont
        \interlinepenalty\@M
        \Huge \bfseries\thechapter\space  #1\par\nobreak
        \vskip 20\p@
        }}
\makeatother

% Document
\begin{document}

    \title{\includegraphics[width=10cm]{logo.JPEG} \\ \textcolor{Goldenrod}{\textbf{Università degli Studi di Salerno}} \\
    Documentazione progetto\\ Fondamenti di Intelligenza Artificiale \\ a.a. 2022/2023 \\ prof. Fabio Palomba}
    \author{
        \begin{tabular}{p{5cm}l}
            \textit{Autore} & \textit{Matricola}\\
            \hline
            Costante Luigina & 0512110457\\
            Lo Conte Simona & 0512110922\\
            Napolillo Marta & 0512109836 \\
        \end{tabular}
    }
    \date{}
    \maketitle

    \tableofcontents

    \chapter{Introduzione}\label{ch:introduzione}

    Con l'avvento del digitale è in costante crescita il numero di persone - di ogni fascia di età - che scelgono di guardare un film nel proprio
    tempo libero. Di conseguenza, aumenta la voglia di scoprire sempre nuovi contenuti in base alle proprie preferenze e rimanere costantemente
    aggiornati sulle ultime novità. La maggior parte degli utenti, però, è spesso indecisa su quale film scegliere e passa gran parte del tempo
    a navigare tra i contenuti disponibili. A tal proposito \textit{iLike}, oltre a realizzare una piattaforma unificata che consente di recensire
    contenuti, offre la possibilità di interagire con un Conversational Agent, la quale permette di visualizzare i film richiesti dall'utente,
    personalizzati sulla base delle sue preferenze. Il Conversational Agent è di fondamentale importanza poichè permette agli utenti indecisi di
    ricevere consigli personalizzati evitando di passare ore ed ore nella scelta di un film.


    \chapter{Business Understanding}\label{ch:business-understanding}


        \section{Obiettivi di business}\label{sec:obiettivi-di-business}

            L'obiettivo principale di \textit{iLike} è la realizzazione di un Conversational Agent, che permetterà all'utente di interagirvi per
            richiedere consigli su film da guardare. Lo scopo è quello di consentire una facile interazione degli utenti con la nostra applicazione,
            permettendo a utenti indecisi di ricevere consigli personalizzati su film che potrebbero essere di interesse personale, in base a quelli
            appartenenti alle liste personali create oppure in base ad un genere scelto dall'utente.

        \section{PEAS}\label{sec:peas}

            Specifica PEAS dell'ambiente.

                \paragraph{}

                \begin{tabular}{|>{\columncolor{Goldenrod}}c|p{10cm}|}
                    \hline \textbf{Performance} & Capacità dell’agente di suggerire all’utente film che rispecchiano i suoi gusti. \\
                    \hline \textbf{Environment} & L’ambiente in cui l’agente opera è rappresentato da iLike, un’applicazione in cui gli
                    utenti possono scrivere recensioni ed esprimere preferenze sui contenuti che si trovano all’interno di essa.\\
                    \hline \textbf{Actuators} & Risposta del Conversational Agent.\\
                    \hline \textbf{Sensors} & Utterances (messaggi in linguaggio naturale dati in input al CA da un utente umano).\\
                    \hline
                \end{tabular}


        \section{Proprietà dell'ambiente}\label{sec:proprieta-dell'ambiente}
            L’ambiente possiede le seguenti proprietà:
                \begin{itemize}
                    \item \textbf{Completamente osservabile}: l’agente ha accesso all’elenco dei contenuti presenti nell’applicazione
                    e alle preferenze degli utenti in qualsiasi momento;
                    \item \textbf{Stocastico}: lo stato dell’ambiente varia indipendentemente dall’azione intrapresa dall’agente;
                    \item \textbf{Sequenziale}: le decisioni prese dall’agente dipendono dalle azioni passate dell’utente;
                    \item \textbf{Statico}: nel momento in cui l’agente sta elaborando la sua decisione l’utente non può modificare
                    le sue preferenze;
                    \item \textbf{Discreto}: i suggerimenti dati dall’agente dipendono dalla combinazione di contenuti preferiti di cui
                    l’utente dispone o da un genere stabilito ed esistono un numero limitato di possibili combinazioni;
                    \item \textbf{Singolo-agente}: esiste un unico agente che opera nell’ambiente.
                \end{itemize}

        \section{Analisi del problema}\label{sec:analisi-del-problema}

        Il problema che l'agente intelligente dovrà risolvere consiste nel suggerire film da vedere in base ai contenuti
        presenti nei dataset dell'applicazione e soprattutto in merito alle preferenze espresse dagli utenti (in base ai
        contenuti delle liste personali o ad un genere scelto).
        Il problema in esame può essere risolto con un algoritmo di apprendimento in quanto consiste nel migliorare l'esecuzione
        di un task (T=fornire suggerimenti personalizzati) rispetto ad una misura di prestazione (P= numero di suggerimenti accettati
        dall'utente) e sulla base dell'esperienza (E= database di contenuti non etichettati). Inoltre l'algoritmo di apprendimento
        in questione è di tipo non supervisionato in quanto non si dispone di un database contenente dati già etichettati, bensì dovrà
        essere l'agente capace di apprendere il valore reale della variabile dipendente sulla base delle conoscenze di cui dispone.
        Nello specifico il problema in esame può essere risolto tramite l'utilizzo di un algoritmo di clustering. Una volta che l'utente
        ha espresso le sue preferenze riguardanti contenuti presenti nell'applicazione, l'algoritmo creerà, in base ad una misura di similarità
        (che verrà definita in seguito), dei cluster contenenti film dotati di un certo grado di omogeneità. Procederà quindi a consigliare nuovi
        film in base alla clusterizzazione effettuata.

        I suggerimenti verranno dati solo qualora l'utente ne faccia richiesta ed il tutto avviene in maniera automatica tramite l'utilizzo
        di un Conversational Agent.

    \chapter{Data Understanding}\label{ch:data-understanding}

        \section{Acquisizione dei dataset}\label{sec:acquisizione-dei-dataset}
            Durante la scelta dai dati da fornire al machine learning le possibili scelte da seguire erano sostanzialmente due:
            \begin{itemize}
                \item Creare un dataset contenente gli utenti di iLike ed analizzare il loro comportamento, al fine di creare
                      cluster di utenti i quali hanno preferenze similari;
                \item Cercare dataset con le informazioni relative ai film e creare cluster di film.
            \end{itemize}

            I problemi riscontrati sono:
            \begin{itemize}
                \item La disponibilità di dati era maggiore nei dataset già esistenti;
                \item Ogni utente ha gusti differenti, quindi la similarità tra utenti, rappresentata come il numero di contenuti uguali
                      appartente alle proprie liste, può non essere sempre veritera;
                \item Individuare dataset con un numero ottimale di istanze e le giuste informazioni sui film richiede
                      un'accurata analisi.
            \end{itemize}
            Al seguito di un trade-off tra le due alternative abbiamo preferito utilizzare dataset già esistenti relativi ai film,
            poichè la disponibilità di dati e la giusta similarità di elementi in un cluster agevola le prestazioni dell'algoritmo
            di machine learning.

        \section{Analisi dei dataset}\label{sec:analisi-dei-dataset}
            Il dataset utilizzato riguardo i \href{https://www.kaggle.com/datasets/stefanoleone992/filmtv-movies-dataset?resource=download}{\underline{Film}}
            è reperibile sulla piattaforma Kaggle.


    \chapter{Data Preparation}\label{ch:data-preparation}
        In fase di Data Understanding abbiamo scelto l'utilizzo di un dataset già esistente, quindi è opportuno effettuare Data Preparation.
        Lo scopo di questa fase è pulire i dati al fine di passarli all'algoritmo di Machine Learning.

        \section{Data Cleaning}\label{sec:data-cleaning}
            Lo scopo del Data Cleaning è gestire dati rumorosi e/o nulli.
            A tal proposito, è stata effettuata inizialmente un'analisi dei dati al fine di evidenziare dati rumorosi.
            Si è notato che le colonne 'voto\_medio' e 'voti\_totali' rispecchiano esattamente la media e la somma delle colonne
            'voto\_critica' e 'voto\_pubblico'. Dunque abbiamo deciso di elimnare le colonne 'voto\_critica' e 'voto\_pubblico' e
            non eliminare le colonne 'voto\_medio' e 'voti\_totali', rimandando tale scelta nel Feature Selection, per valutare
            quale delle due ha una maggiore corelazione tra le altre variabili.

            In seguito è riportata una tabella contenente il nome della colonna, il numero di elementi mancanti e la scelta di
            Data Imputation effettuata.\\


            \begin{tabular}{ |c|c|c| }
                \hline \rowcolor{Goldenrod} Nome Colonne & Numero Elementi & Scelta Data Imputation \\
                \hline genere & 95 & Eliminazione Riga \\
                \hline paese & 11 & Eliminazione Riga \\
                \hline registi & 33 & Eliminazione Riga \\
                \hline attori & 2.027 & Eliminazione Colonna \\
                \hline descrizione & 1.449 & Eliminazione Riga \\
                \hline note & 21.628 & Eliminazione Colonna \\
                \hline
            \end{tabular}
            \\

            Le scelte sopra riportate sono state effettuate per evitare di eliminare un eccessivo numero di righe.
            Si fa eccezione per la colonna descrizione in quanto tale colonna è necessaria all'interno dell'applicazione
            iLike, per altre funzionalità non riguardanti il modulo di Intelligenza Artificiale.

        \section{Feature Scaling}\label{sec:feature-scaling}
            Lo scopo del Feature Scaling è normalizzare i valori numeri del dataset, portandoli tutti nello stesso range.
            La tabella sotto riportata confronta il range iniziale e i range ottenuti prima con la tecnica del Min-Max normalizzation
            e in seguito con lo Z-Score Normalizzation.

            \begin{tabular}{ |c|c|c|c| }
                \hline \rowcolor{Goldenrod} Nome Colonne & Range Iniziale & Media Min-Max & Media Z-Score \\
                \hline erotismo & 0-4 & 0.076504 &  -0.354508 \\
                \hline tensione & 0-5 & 0.188511 & -0.354313 \\
                \hline impegno & 0-5 & 0.139378 & -0.354420 \\
                \hline ritmo & 0-5 & 0.278423 & -0.354193 \\
                \hline humor & 0-5 & 0.120133 & -0.354415 \\
                \hline voti\_totali & 1-1052 & 0.035554 & -0.345445 \\
                \hline voto\_medio & 1-10 & 0.531023 & -0.353328 \\
                \hline durata & 41-924 & 0.067289 & -0.333306 \\
                \hline anno & 1911-2023 & 0.741849 & 0.061447 \\
                \hline
            \end{tabular}

        \section{Feature Selection}\label{sec:feature-selection}


        \section{Data Balancing}\label{sec:data-balancing}


    \chapter{Modeling}\label{ch:modeling}


        \section{Scelta dell'algoritmo da utilizzare}\label{sec:scelta-dell'algoritmo-da-utilizzare}


        \section{Fase di addestramento}\label{sec:fase-di-addestramento}


    \chapter{Evaluation}\label{ch:evaluation}


        \section{Elbow point}\label{sec:elbow-point}


        \section{Silhouette coefficient}\label{sec:silhouette-coefficient}


        \section{MoJo distance}\label{sec:mojo-distance}


    \chapter{Deployment}\label{ch:deployment}


    \chapter{Glossario}\label{ch:glossario}
        \begin{adjustbox}{width=\columnwidth,center}
            \begin{tabular}{|>{\columncolor{Goldenrod}}c|p{10cm}|}
                \hline \textbf{Conversational Agent/CA} & É un bot che interpreta e risponde alle dichiarazioni fatte dagli utenti
                in un linguaggio naturale, attraverso la  generazione di una conversazione simil-umana.\\
                \hline \textbf{Lista di contenuti} & Sottoinsieme di contenuti offerti da iLike, scelti dagli utenti secondo i loro gusti
                e inseriti nelle proprie liste disponibili sul proprio profilo personale.\\
                \hline \textbf{Cluster} & Sottoinsieme di contenuti con caratteristiche simili.\\
                \hline \textbf{Machine Learning} & É la branca dell'Intelligenza Artificiale che include tutti gli algoritmi
                        che possano imparare dai dati e sulla base di questi fare previsioni.\\
                \hline \textbf{Data Imputation} & É l'insieme di tecniche che possono stimare il valore di dati mancanti
                        sulla base dei dati disponibili oppure mitigare il problema dei dati mancanti.\\
                \hline
            \end{tabular}
        \end{adjustbox}


\end{document}