
% Preamble
\documentclass[a4paper, 12pt]{report}
\usepackage[T1]{fontenc}
\usepackage[utf8]{inputenc}
\usepackage[italian]{babel}
\usepackage{lipsum}
\usepackage{url}
\usepackage{array}
\usepackage{color}
\usepackage{colortbl}
\usepackage[dvipsnames]{xcolor}
\usepackage{graphicx}
\usepackage[colorlinks = true,
    linkcolor = darkgray,
    urlcolor  = darkgray,
    citecolor = white,
    anchorcolor = darkgray]{hyperref}

\makeatletter
    \def\@makechapterhead#1{%
        \vspace*{1\p@}%
        {\parindent \z@ \raggedright
        \normalfont
        \interlinepenalty\@M
        \Huge \bfseries\thechapter\space  #1\par\nobreak
        \vskip 20\p@
        }}
\makeatother

% Document
\begin{document}

    \title{\includegraphics[width=10cm]{logo.JPEG} \\ \textcolor{Goldenrod}{\textbf{Università degli Studi di Salerno}} \\
    Documentazione progetto Fondamenti di Intelligenza Artificiale \\ a.a. 2022/2023 \\ prof. Fabio Palomba}
    \author{
        \begin{tabular}{p{5cm}l}
            \textit{Autore} & \textit{Matricola}\\
            \hline
            Costante Luigina & 0512110457\\
            Lo Conte Simona & 0512110922\\
            Napolillo Marta & 0512109836 \\
        \end{tabular}
    }
    \date{}
    \maketitle

    \tableofcontents

    \chapter{Introduzione}\label{ch:introduzione}

    Con l'avvento del digitale è in costante crescita il numero di persone - di ogni fascia di età - che leggono libri, ascoltano musica,
    guardano film e serie TV. Di conseguenza, aumenta la voglia di scoprire sempre nuovi contenuti in base alle proprie preferenze e rimanere
    costantemente aggiornati sulle ultime novità. La maggior parte degli utenti,però, è spesso indecisa sui contenuti da scegliere e passa gran parte
    del tempo a navigare tra i contenuti disponibili.
    A tal proposito \textit{iLike}, oltre a realizzare una piattaforma unificata che consente di recensire film, serie TV, libri e album musicali,
    offre la possibilità di interagire con una chat bot, la quale permette di visualizzare i contenuti richiesti dall’utente,personalizzati sulla base
    delle sue preferenze. La chat bot è di fondamentale importanza poichè permette agli utenti indecisi di ricevere consigli personalizzati ed evitare
    di passare ore ed ore nella scelta di un contenuto.



    \chapter{Business Understanding}\label{ch:business-understanding}


        \section{Obiettivi di business}\label{sec:obiettivi-di-business}

        L’obiettivo principale di \textit{iLike} è la realizzazione di una chat bot, che permetterà all’utente di interagirvi
        per richiedere consigli su contenuti che ricadono nei suoi gusti personali.
        Lo scopo è quello di consentire una facile interazione degli utenti con la nostra applicazione, permettendo a utenti indecisi di ricevere
        consigli personalizzati su contenuti che potrebbero essere di interesse personale,in base a quelli appartenenti alle liste personali create
        e sulla base della valutazione inserita all'interno delle recensioni scritte riguardo a determinati contenuti.

    \section{PEAS}\label{sec:peas}

            Specifica PEAS dell'ambiente.

                \paragraph{}

                \begin{tabular}{|>{\columncolor{Goldenrod}}c|p{10cm}|}
                    \hline
                    \textbf{Performance} & Capacità dell’agente di suggerire all’utente contenuti che rispecchiano i suoi gusti. \\
                    \hline
                    \textbf{Environment} & L’ambiente in cui l’agente opera è rappresentato da iLike, un’applicazione in cui gli
                    utenti possono scrivere recensioni ed esprimere preferenze sui contenuti che si trovano all’interno di essa.\\
                    \hline
                    \textbf{Actuators} & Risposta del Conversational Agent.\\
                    \hline
                    \textbf{Sensors} & Utterances (messaggi in linguaggio naturale dati in input al CA da un utente umano).\\
                    \hline
                \end{tabular}


        \section{Proprietà dell'ambiente}\label{sec:proprieta-dell'ambiente}
            L’ambiente possiede le seguenti proprietà:
                \begin{itemize}
                    \item \textbf{Completamente osservabile}: l’agente ha accesso all’elenco dei contenuti presenti nell’applicazione
                    e alle preferenze degli utenti in qualsiasi momento;
                    \item \textbf{Stocastico}: lo stato dell’ambiente varia indipendentemente dall’azione intrapresa dall’agente;
                    \item \textbf{Sequenziale}: le decisioni prese dall’agente dipendono dalle azioni passate dell’utente;
                    \item \textbf{Statico}: nel momento in cui l’agente sta elaborando la sua decisione l’utente non può modificare
                    le sue preferenze;
                    \item \textbf{Discreto}: i suggerimenti dati dall’agente dipendono dalla combinazione di contenuti preferiti di cui
                    l’utente dispone ed esistono un numero limitato di possibili combinazioni;
                    \item \textbf{Singolo-agente}: esiste un unico agente che opera nell’ambiente.
                \end{itemize}

        \section{Analisi del problema}\label{sec:analisi-del-problema}
            Il problema che l’agente intelligente dovrà risolvere consiste nel suggerire film/serie TV da vedere,
            libri da leggere ed album musicali da ascoltare in base ai contenuti presenti nei dataset dell’applicazione
            e soprattutto in base alle preferenze espresse dagli utenti.

            Il problema in esame può essere risolto con un algoritmo di apprendimento in quanto consiste nel migliorare
            l’esecuzione di un task (T=fornire suggerimenti personalizzati) rispetto ad una misura di prestazione
            (P= numero di suggerimenti accettati dall’utente) e sulla base dell’esperienza (E= database di contenuti non
            etichettati).
            Inoltre l’algoritmo di apprendimento in questione è di tipo non supervisionato in quanto non si dispone di un
            database contenente dati già etichettati, bensì dovrà essere l’agente capace di apprendere il valore reale
            della variabile dipendente sulla base delle conoscenze di cui dispone.

            Nello specifico il problema in esame può essere risolto tramite l’utilizzo di un algoritmo di clustering.
            Una volta che l’utente ha espresso le sue preferenze riguardanti contenuti presenti nell’applicazione, l’algoritmo
            creerà, in base ad una misura di similarità (che verrà definita in seguito), dei cluster contenenti film, serie TV,
            libri e album musicali dotati di un certo grado di omogeneità.
            Procederà quindi a consigliare nuovi contenuti in base alla clusterizzazione effettuata.

            I suggerimenti verranno dati solo qualora l’utente ne faccia richiesta ed il tutto avviene in maniera
            automatica tramite l’utilizzo di un Conversational Agent.

    \chapter{Data Understanding}\label{ch:data-understanding}

        \section{Acquisizione dei dataset}\label{sec:acquisizione-dei-dataset}
            Durante la scelta dai dati da fornire al machine learning le possibili scelte da seguire erano sostanzialmente due:
            \begin{itemize}
                \item Creare un dataset contenente gli utenti di iLike ed analizzare il loro comportamento, al fine di creare
                      cluster di utenti i quali hanno preferenze similari;
                \item Cercare dataset con le informazioni sui contenuti e creare cluster di contenuti.
            \end{itemize}

            I problemi riscontrati sono:
            \begin{itemize}
                \item La disponibilità di dati era maggiore nei dataset già esistenti;
                \item Ogni utente ha gusti differenti, quindi la similarità dei contenuti in un cluster può non essere sempre
                      veritera;
                \item Individuare dataset con un numero ottimale di istanze e le giuste informazioni sui contenuti richiede
                      un'accurata analisi.
            \end{itemize}
            Al seguito di un trade-off tra le due alternative abbiamo preferito utilizzare dataset già esistenti relativi ai contenuti,
            poichè la disponibilità di dati e la giusta similarità di elementi in un cluster agevola le prestazioni dell'algoritmo
            di machine learning.

        \section{Analisi dei dataset}\label{sec:analisi-dei-dataset}
            I dataset reperiti sulla piattaforma Kaggle sono:
            \begin{itemize}
                \item \href{https://www.kaggle.com/datasets/stefanoleone992/filmtv-movies-dataset?resource=download}{\underline{Film}}
                \item \href{https://www.kaggle.com/datasets/amritvirsinghx/web-series-ultimate-edition}{\underline{Serie TV}}
                \item \href{https://www.kaggle.com/datasets/lucascantu/top-5000-albums-of-all-time-spotify-features}{\underline{Album Musicali}}
                \item \href{https://www.kaggle.com/datasets/mdhamani/goodreads-books-100k}{\underline{Libri}}
            \end{itemize}



    \chapter{Data Preparation}\label{ch:data-preparation}


        \section{Data Cleaning}\label{sec:data-cleaning}


        \section{Feature Scaling}\label{sec:feature-scaling}


        \section{Feature Selection}\label{sec:feature-selection}


        \section{Data Balancing}\label{sec:data-balancing}


    \chapter{Modeling}\label{ch:modeling}


        \section{Scelta dell'algoritmo da utilizzare}\label{sec:scelta-dell'algoritmo-da-utilizzare}


        \section{Fase di addestramento}\label{sec:fase-di-addestramento}


    \chapter{Evaluation}\label{ch:evaluation}


        \section{Elbow point}\label{sec:elbow-point}


        \section{Silhouette coefficient}\label{sec:silhouette-coefficient}


        \section{MoJo distance}\label{sec:mojo-distance}


    \chapter{Deployment}\label{ch:deployment}


    \chapter{Glossario}\label{ch:glossario}
        \begin{tabular}{|>{\columncolor{Goldenrod}}c|p{10cm}|}
            \hline
            \textbf{Contenuto} & Elemento appartenente all’insieme di film, serie TV, libri e album musicali.\\
            \hline
            \hline
            \textbf{Cluster} & Sottoinsieme di contenuti con caratteristiche simili.\\
            \hline
            \hline
            \textbf{Machine Learning} & É la branca dell'Intelligenza Artificiale che include tutti gli algoritmi
                    che possano imparare dai dati e sulla base di questi fare previsioni.\\
            \hline
        \end{tabular}

\end{document}